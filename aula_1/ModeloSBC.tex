% ----------------------------------------------------------------
% Customization by Argemiro Oliveira <arge.unb@gmail.com>
% ----------------------------------------------------------------
\documentclass[12pt]{article}
\usepackage{ifpe-template}
\usepackage{tikz}
\usepackage{graphicx,url}
\usepackage[brazil]{babel}
\usepackage[utf8x]{inputenc}  
\pagestyle{plain} 
\usepackage{graphicx,url}
\usepackage[utf8x]{inputenc}
\usepackage[brazil]{babel}
\usepackage{comment}
\usepackage{amsmath}
\usepackage{amssymb}
\usepackage{enumerate}
\usepackage{subcaption}
\usepackage[backref=page]{hyperref}
\hypersetup{
    colorlinks=true,
    allcolors=blue,
}

\pagenumbering{arabic} 

%%%%%%%%%%%%%%%% ALTERAR %%%%%%%%%%%%%%%%
\newcommand{\campus}{Curitiba PR}
\newcommand{\curso}{Fundamentos de Algorıtmos e Estrutura de Dados - PUC PR}
\newcommand{\dataaprovacao}{02/08/2024}
%%%%%%%%%%%%%%%% ALTERAR %%%%%%%%%%%%%%%%


%%%%%%%%%%%%%%%% NÃO ALTERAR %%%%%%%%%%%%%%%%
%configurando referências de acordo com a ABNT
\usepackage[alf]{abntex2cite}
\usepackage{fancyhdr}
\fancyhead{} % clear all header fields
\renewcommand{\headrulewidth}{0pt} % no line in header area
\fancyfoot{} % clear all footer fields
\fancyfoot[R]{ \vspace{2pt} \footnotesize \thepage}           % page number in "outer" position of footer line
%\fancyfoot[RE,LO]{Instituto Federal de Pernambuco. \textit{Campus} \campus. Curso de \curso. \dataaprovacao. } % 

\fancypagestyle{firststyle}
{
   \fancyfoot[L]{Pontifıcia Universidade Catolica do Parana (PUC). \textit{Campus} \campus. Curso de \curso. \dataaprovacao.} % 
}
\pagestyle{firststyle}

\sloppy
%%%%%%%%%%%%%%%% NÃO ALTERAR %%%%%%%%%%%%%%%%




\title{Trabalho 1 - Comparativo entre Métodos de Ordenação}
%\entitle{TÍTULO EM INGLÊS}


\author{Argemiro O. Neto\inst{1}, David S. Siqueira\inst{2}, Jean P. S. Filho\inst{3} }


\address{
Fundamentos de Algorítmos e Estrutura de Dados - PUC PR
%\nextinstitute
  %Department of Computer Science -- University of Durham\\
  %Durham, U.K.
\nextinstitute
  Programa de Pós Graduação em Informática Aplicada (PPGIA)\\
  Pontifícia Universidade Católica do Paraná (PUC)\\ 
  Curitiba - PR - Brasil
  \email{arge.unb@gmail.com, dave.sathler@gmail.com,
  jomi@inf.furb.br}
}

\begin{document} 

\maketitle
\thispagestyle{firststyle}

\begin{tikzpicture}[overlay, remember picture]
  \node[xshift=-3.4cm,yshift=-2cm] at (current page.north east) {\includegraphics[width=3.8cm,height=2cm]{logo PPGIA.png}};
\end{tikzpicture}



\hrule

\begin{resumo} 
  Este meta-artigo descreve o estilo a ser usado na confecção de artigos e
  resumos de artigos para publicação nos anais das conferências organizadas
  pela SBC. É solicitada a escrita de resumo e abstract apenas para os artigos
  escritos em português. Artigos em inglês deverão apresentar apenas abstract.
  Nos dois casos, o autor deve tomar cuidado para que o resumo (e o abstract)
  não ultrapassem 10 linhas cada, sendo que ambos devem estar na primeira
  página do artigo.
  
  \textbf{Palavras-chave: }
\end{resumo}


\begin{abstract}
  This meta-paper describes the style to be used in articles and short papers
  for SBC conferences. For papers in English, you should add just an abstract
  while for the papers in Portuguese, we also ask for an abstract in
  Portuguese (``resumo''). In both cases, abstracts should not have more than  10 lines and must be in the first page of the paper.
  
  \textbf{Keywords: }
\end{abstract}


\hrule
     
\section{Introdução}

........................................

\section{Metodologia Utilizada} \label{sec:firstpage}

................................

\section{Resultados Obtidos}\label{sec:sections-paragraphs}

.........................

\section{Discussão dos Resultados}\label{sec:figs}

...................

\section{Conclusão do Estudo}

..................
%abaixo o link para o arquivo bibtex
\bibliography{ifpe-template}

\end{document}
